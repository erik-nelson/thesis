\begin{abstract}
  This thesis proposes compressing a mobile robot's map as a means of increasing the speed
  at which it is able to explore an unknown environment. Exploration is a useful capability
  for autonomous mobile robots that must operate outside of controlled factories
  and laboratories. Recent advances in exploration employ techniques
  that compute control actions by analyzing information-theoretic metrics on the
  robot's map. Information-theoretic metrics are generally computationally expensive to
  evaluate, ultimately limiting the speed at which a robot is able to explore.

  To reduce the computational cost of exploration, this thesis develops an information-theoretic
  strategy for compressing a robot's map, in turn
  allowing information-based reward to be evaluated more efficiently. To remain
  effective for exploration, this strategy must compress maps in a way that
  sacrifices a minimal amount of information about expected future sensor
  measurements. Adaptively compressing the robot's map to different resolutions
  in response to local environment complexity, and propagating the efficiency
  gains through to planning frequency and velocity gives rise to intelligent
  behaviors such as speeding up in open expanses. These methods are used to
  demonstrate information-theoretic exploration through mazes and cluttered
  indoor environments at speeds of 3 m/s in simulation, and 1.6 m/s on a ground
  robot.
\end{abstract}
