\chapter{Introduction}

Robots are emerging from controlled factories and
laboratories into our homes, workplaces, roads, and public airspaces.
Alongside their transition into these unstructured and transient environments
comes their need to be able to explore, characterize, and catalog their surroundings.
Mobile robot autonomy is generally accomplished by referring to a map - a 2D or 3D
probabilistic representation of the locations of obstacles in the workspace.
With access to a map, robots can localize to determine their position, plan collision-free
trajectories to goals, locate objects for interaction, and make decisions by
reasoning about the geometry and dynamics of the world. Given that a robot's map
is of critical importance for most autonomy tasks, robots that find
themselves initialized without a priori access to a map should be capable of
autonomously, efficiently, and intelligently creating one.

The exercise of choosing and executing actions that lead a robot to learn more about its own
map is known as \textit{active perception} or \textit{exploration}, and
is the central topic of this thesis. Active perception has previously been studied with a
multitude of sensor models, environment representations, and robot dynamics
models. The active perception task itself can be split into two
components~\cite{shen20113d}:

\begin{enumerate}
  \item Identifying regions in the environment that, when visited, will
    spatially extend or reduce uncertainty in the current map
  \item Autonomously navigating to the aforementioned regions, while
    simultaneously localizing to the map and updating it with acquired sensor
    measurements
\end{enumerate}

A motivating example is depicted in Fig.~\ref{fig:motivation}, where a household
service robot is initialized in an unknown environment. Prior to accomplishing
tasks that a human might ask it to perform, the robot must learn its
surroundings and build a map of the house. Ideally this phase of
initialization would be fast, as it is a prerequisite to the main functionality
of the robot, and also might be required when furniture is moved or
household objects are displaced. Where should the robot travel to observe the
most of the environment in the shortest amount of time? Virtually any autonomous robot
operating in an uncontrolled environment will require a map-building
initialization phase, motivating algorithms for high-speed and intelligent
exploration.

This thesis introduces an assortment of information-theoretic optimizations that
increase the efficiency of active perception when using a beam-based sensor
model (e.g. LIDAR, time-of-flight cameras, structured light sensors) and
an occupancy grid map~\cite{elfes1989using}. Applying these optimizations during exploration allows a
robot to consider a significantly larger number of future locations to move
towards in its partially observed environment, regardless of the planning
strategy used. Additionally, this thesis presents a method for analyzing the
complexity of the local environment and adapting the robot's map resolution,
planning frequency, movement speed, and exploration behaviors accordingly. By
adapting these parameters online, an exploring robot is able to

%allow a robot to analyze the complexity and structure of its local surroundings
%Novel to this thesis is an assortment of information-theoretic
%optimizations that allow a robot to analyze the complexity of its surroundings
%online and change its map representation to increase both the speed and
%efficiency of exploration.

\begin{figure}[ht]
  \centering
  \includegraphics[width=0.6\textwidth]{example-image.png}
  \caption{A household service robot awakes in an unknown environment. Prior to
  accomplishing its main functionalities, it will require a map of its surroundings.
What sequence of actions should it take to minimize the amount of time spent
exploring? \label{fig:motivation}}
\end{figure}

\section{Previous Work}

Prior approaches to mobile robot active perception fall into two
broad categories: \textit{geometric} approaches that reason about the locations and
presence of obstacles and
free space in the robot's map\todo{~\cite{}},
and more recently, \textit{information-theoretic} approaches that treat the map as
a multivariate random variable and choose actions that will maximally reduce
its uncertainty\todo{~\cite{}}.
Both categories of approaches solve the first
enumerated item in the above list, and assume that a planner and Simultaneous
Localization and Mapping (SLAM) framework are available to accomplish the second
item.

Many successful geometric exploration approaches build upon the seminal work of
Yamauchi~\cite{yamauchi1997frontier}, guiding the robot to \textit{frontiers} - regions on the boundary
between free and unexplored space in the map (Fig.~\ref{fig:frontiers}).
As multiple frontiers often exist simultaneously in a partially explored map, a
variety of heuristics and spatial metrics can be used to decide which frontier to
travel towards~\cite{lavalle2006planning}. For example, an agent may decide to
visit the frontier whose path through the configuration space from the robot's current
position has minimum length, or requires minimal time or energy input to
traverse. Similarly, an agent may decide to only plan paths to locations
from which frontiers can be observed by its onboard sensors.

\begin{figure}
  \centering
  \begin{subfigure}[t]{0.45\textwidth}
    \centering
    \includegraphics[width=0.9\textwidth]{example-image}
    \caption{\todo{A geometric approach is used to identify frontiers,
    or the intersections between free and occupied cells in the robot's map.}\label{fig:frontiers}}
  \end{subfigure}
  \hfill
  \begin{subfigure}[t]{0.45\textwidth}
    \centering
    \includegraphics[width=0.9\textwidth]{example-image}
    \caption{\todo{Information-theoretic exploration}\label{fig:info-theoretic}}
  \end{subfigure}
  \caption{A robot is faced with a decision-making problem. Will the red or
    green path lead to a higher future reward? \label{fig:exploration_strategies}}
\end{figure}

Other geometric exploration algorithms include

While effective in 2D environments, geometric exploration algorithms have
several restrictive qualities. First, the na\"{i}ve extension of frontier exploration
from 2D to 3D poses a non-trivial challenge~\cite{shen20113d}; as the
dimensionality of the workspace increases, frontiers are more frequently identified in
locations that do not truly represent boundaries between free and occupied
space.


\section{Thesis Problem}

% Thesis Problem
\begin{center} \fbox{
  \parbox{0.9\linewidth}
  { {\bf Thesis Problem:} Solutions to the mobile robot active perception task
  that involve optimization of information-theoretic cost functions are too
  computationally expensive for high-speed exploration in complex environments.}
} \end{center}

% Thesis Statement
%\begin{center} \fbox{
%  \parbox{0.9\linewidth}
%  { {\bf Thesis Statement:} Occupancy grid maps can be compressed in such a way
%  that information metrics computed between the map and range sensor
%measurements are minimally affected,}
%} \end{center}



\section{Thesis Outline}
